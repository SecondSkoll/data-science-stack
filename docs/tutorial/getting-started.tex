\documentclass[a4paper]{article}
% generated by Docutils <https://docutils.sourceforge.io/>
\usepackage{cmap} % fix search and cut-and-paste in Acrobat
\usepackage{ifthen}
\usepackage[T1]{fontenc}
\usepackage{alltt}
\usepackage{color}
\setcounter{secnumdepth}{0}

%%% Custom LaTeX preamble
% PDF Standard Fonts
\usepackage{mathptmx} % Times
\usepackage[scaled=.90]{helvet}
\usepackage{courier}

%%% User specified packages and stylesheets

%%% Fallback definitions for Docutils-specific commands

% class handling for environments (block-level elements)
% \begin{DUclass}{spam} tries \DUCLASSspam and
% \end{DUclass}{spam} tries \endDUCLASSspam
\ifx\DUclass\undefined % poor man's "provideenvironment"
 \newenvironment{DUclass}[1]%
  {% "#1" does not work in end-part of environment.
   \def\DocutilsClassFunctionName{DUCLASS#1}
     \csname \DocutilsClassFunctionName \endcsname}%
  {\csname end\DocutilsClassFunctionName \endcsname}%
\fi

% admonition environment (specially marked topic)
\ifx\DUadmonition\undefined % poor man's "provideenvironment"
 \newbox{\DUadmonitionbox}
 \newenvironment{DUadmonition}%
  {\begin{center}
     \begin{lrbox}{\DUadmonitionbox}
       \begin{minipage}{0.9\linewidth}
  }%
  {    \end{minipage}
     \end{lrbox}
     \fbox{\usebox{\DUadmonitionbox}}
   \end{center}
  }
\fi

% basic code highlight:
\providecommand*\DUrolecomment[1]{\textcolor[rgb]{0.40,0.40,0.40}{#1}}
\providecommand*\DUroledeleted[1]{\textcolor[rgb]{0.40,0.40,0.40}{#1}}
\providecommand*\DUrolekeyword[1]{\textbf{#1}}
\providecommand*\DUrolestring[1]{\textit{#1}}

% custom inline roles: \DUrole{#1}{#2} tries \DUrole#1{#2}
\providecommand*{\DUrole}[2]{%
  \ifcsname DUrole#1\endcsname%
    \csname DUrole#1\endcsname{#2}%
  \else%
    #2%
  \fi%
}

% title for topics, admonitions, unsupported section levels, and sidebar
\providecommand*{\DUtitle}[1]{%
  \smallskip\noindent\textbf{#1}\smallskip}

% titlereference standard role
\providecommand*{\DUroletitlereference}[1]{\textsl{#1}}

% hyperlinks:
\ifthenelse{\isundefined{\hypersetup}}{
  \usepackage[colorlinks=true,linkcolor=blue,urlcolor=blue]{hyperref}
  \usepackage{bookmark}
  \urlstyle{same} % normal text font (alternatives: tt, rm, sf)
}{}
\hypersetup{
  pdftitle={Getting Started},
}

%%% Body
\begin{document}
\title{Getting Started%
  \label{getting-started}%
  \label{tutorial}}
\author{}
\date{}
\maketitle

This guide will walk you through how you can get started with the Data
Science Stack. From setting up MicroK8s in your host environment and
configuring GPU drivers, all the way to running your first notebook.


\section{Prerequisites%
  \label{prerequisites}%
}

\begin{itemize}
\item Ubuntu 22.04

\item An internet connection

\item DSS relies on %
\raisebox{1em}{\hypertarget{problematic-5}{}}\hyperlink{system-message-5}{\textbf{\color{red}`MicroK8s`\_}}, that requires as little as 540MB of memory.
But to accommodate workloads, we recommend a system with at least 20G
of disk space and 4G of memory.
\end{itemize}


\section{Setting up MicroK8s%
  \label{setting-up-microk8s}%
}

The DSS relies on a container orchestration system, that can also take
care of exposing the host GPUs to the workloads. In this case we will use
MicroK8s snap, which gets installed on the host machine.

All the workloads and state managed by DSS will be running on top of
MicroK8s.

You can install MicroK8s with the following commands:

\begin{DUclass}{code}
\begin{DUclass}{bash}
\begin{quote}
\ttfamily\raggedright
sudo\DUrole{whitespace}{~}snap\DUrole{whitespace}{~}install\DUrole{whitespace}{~}microk8s\DUrole{whitespace}{~}-{}-channel\DUrole{whitespace}{~}\DUrole{literal}{\DUrole{number}{1}}.28/stable\DUrole{whitespace}{~}-{}-classic\DUrole{whitespace}{~\\
}sudo\DUrole{whitespace}{~}microk8s\DUrole{whitespace}{~}\DUrole{name}{\DUrole{builtin}{enable}}\DUrole{whitespace}{~}storage\DUrole{whitespace}{~}dns\DUrole{whitespace}{~}rbac
\end{quote}
\end{DUclass}
\end{DUclass}


\section{Install the DSS CLI%
  \label{install-the-dss-cli}%
}

At this point we've installed MicroK8s and configured it to use the host's
NVIDIA GPU. The next step now is to install the DSS CLI snap.

You can install the CLI with the following command:

\begin{DUclass}{code}
\begin{DUclass}{bash}
\begin{quote}
\ttfamily\raggedright
sudo\DUrole{whitespace}{~}snap\DUrole{whitespace}{~}install\DUrole{whitespace}{~}data-science-stack\DUrole{whitespace}{~}-{}-channel\DUrole{whitespace}{~}latest/stable
\end{quote}
\end{DUclass}
\end{DUclass}


\section{Initialise the DSS%
  \label{initialise-the-dss}%
}

Now that you have the DSS CLI installed the next step is to initialise
the DSS on top of MicroK8s and prepare MLFlow.

You can initialise the DSS with the following command:

\begin{DUclass}{code}
\begin{DUclass}{bash}
\begin{quote}
\ttfamily\raggedright
dss\DUrole{whitespace}{~}initialize\DUrole{whitespace}{~}-{}-kubeconfig\DUrole{operator}{=}\DUrole{literal}{\DUrole{string}{\DUrole{double}{\textquotedbl{}}}}\DUrole{keyword}{\$(}sudo\DUrole{whitespace}{~}microk8s\DUrole{whitespace}{~}config\DUrole{keyword}{)}\DUrole{literal}{\DUrole{string}{\DUrole{double}{\textquotedbl{}}}}
\end{quote}
\end{DUclass}
\end{DUclass}

\begin{DUclass}{note}
\begin{DUadmonition}
\DUtitle{Note}

The \DUroletitlereference{dss initialize} command might take a few minutes to complete.
\end{DUadmonition}
\end{DUclass}


\section{Launch a Notebook%
  \label{launch-a-notebook}%
}

At this point the DSS is setup on the host and you are ready to start
managing containerised Notebook environments. You can use the DSS CLI
now to launch containerised Notebook environments and access JupyterLab.

You can start your first Notebook with the following command:

\begin{DUclass}{code}
\begin{DUclass}{bash}
\begin{quote}
\ttfamily\raggedright
dss\DUrole{whitespace}{~}create\DUrole{whitespace}{~}my-tensorflow-notebook\DUrole{whitespace}{~}-{}-image\DUrole{operator}{=}kubeflownotebookswg/jupyter-tensorflow-cuda:v1.8.0
\end{quote}
\end{DUclass}
\end{DUclass}

Once the command succeeds it will also return a URL that can be used
to connect to the JupyterLab UI of that Notebook.
For example you should see output like this:

\begin{DUclass}{system-message}
\begin{DUadmonition}
\DUtitle{system-message
}

{\color{red}WARNING/2} in \texttt{getting-started.rst}, line~80

Cannot analyze code. No Pygments lexer found for \textquotedbl{}none\textquotedbl{}.

\begin{quote}
\begin{alltt}
.. code-block:: none

   [INFO] Executing create command
   [INFO] Waiting for deployment my-tensorflow-notebook in namespace dss to be ready...
   [INFO] Deployment my-tensorflow-notebook in namespace dss is ready
   [INFO] Success: Notebook my-tensorflow-notebook created successfully.
   [INFO] Access the notebook at http://10.152.183.42:80.

\end{alltt}
\end{quote}
backrefs: \end{DUadmonition}
\end{DUclass}


\section{Next Steps%
  \label{next-steps}%
}

\begin{itemize}
\item Want to utilize the NVIDIA GPUs in your machine? See %
\raisebox{1em}{\hypertarget{problematic-1}{}}\hyperlink{system-message-1}{\textbf{\color{red}:ref:`setup-nvidia-drivers`}}

\begin{DUclass}{system-message}
\begin{DUadmonition}
\DUtitle{system-message
\raisebox{1em}{\hypertarget{system-message-1}{}}}

{\color{red}ERROR/3} in \texttt{getting-started.rst}, line~90

\hyperlink{problematic-1}{
Unknown interpreted text role \textquotedbl{}ref\textquotedbl{}.
}\end{DUadmonition}
\end{DUclass}

\item Want to learn how to interact with your Notebooks? Try %
\raisebox{1em}{\hypertarget{problematic-2}{}}\hyperlink{system-message-2}{\textbf{\color{red}:ref:`jupyter-notebooks`}}

\begin{DUclass}{system-message}
\begin{DUadmonition}
\DUtitle{system-message
\raisebox{1em}{\hypertarget{system-message-2}{}}}

{\color{red}ERROR/3} in \texttt{getting-started.rst}, line~91

\hyperlink{problematic-2}{
Unknown interpreted text role \textquotedbl{}ref\textquotedbl{}.
}\end{DUadmonition}
\end{DUclass}

\item Want to learn more about handling data? See %
\raisebox{1em}{\hypertarget{problematic-3}{}}\hyperlink{system-message-3}{\textbf{\color{red}:ref:`access-data`}}

\begin{DUclass}{system-message}
\begin{DUadmonition}
\DUtitle{system-message
\raisebox{1em}{\hypertarget{system-message-3}{}}}

{\color{red}ERROR/3} in \texttt{getting-started.rst}, line~92

\hyperlink{problematic-3}{
Unknown interpreted text role \textquotedbl{}ref\textquotedbl{}.
}\end{DUadmonition}
\end{DUclass}

\item Want to connect to MLflow? See %
\raisebox{1em}{\hypertarget{problematic-4}{}}\hyperlink{system-message-4}{\textbf{\color{red}:ref:`notebook-mlflow`}}

\begin{DUclass}{system-message}
\begin{DUadmonition}
\DUtitle{system-message
\raisebox{1em}{\hypertarget{system-message-4}{}}}

{\color{red}ERROR/3} in \texttt{getting-started.rst}, line~93

\hyperlink{problematic-4}{
Unknown interpreted text role \textquotedbl{}ref\textquotedbl{}.
}\end{DUadmonition}
\end{DUclass}
\end{itemize}


\section[Docutils System Messages]{\color{red}Docutils System Messages%
}

\begin{DUclass}{system-message}
\begin{DUadmonition}
\DUtitle{system-message
\raisebox{1em}{\hypertarget{system-message-5}{}}}

{\color{red}ERROR/3} in \texttt{getting-started.rst}, line~15

\hyperlink{problematic-5}{
Unknown target name: \textquotedbl{}microk8s\textquotedbl{}.
}\end{DUadmonition}
\end{DUclass}

\end{document}
